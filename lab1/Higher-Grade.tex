
%%%%%%%%%%%%%%%%%%%%%%%%%%%%%%%%%%%%%%%%%
% Uppsala University Assignment Title Page 
% LaTeX Template
% Version 1.0 (27/12/12)
%
% This template has been downloaded from:
% http://www.LaTeXTemplates.com
%
% Original author:
% WikiBooks (http://en.wikibooks.org/wiki/LaTeX/Title_Creation)
% Modified by Elsa Slattegard to fit Uppsala university
% License:
% CC BY-NC-SA 3.0 (http://creativecommons.org/licenses/by-nc-sa/3.0/)

%\title{Title page with logo}
%----------------------------------------------------------------------------------------
%	PACKAGES AND OTHER DOCUMENT CONFIGURATIONS
%----------------------------------------------------------------------------------------

\documentclass[12pt]{article}
\usepackage[english]{babel}
\usepackage[utf8x]{inputenc}
\usepackage{amsmath}
\usepackage{graphicx}
\usepackage{float}
\usepackage[colorinlistoftodos]{todonotes}
\usepackage{listings}
\usepackage{color}

\definecolor{code_bg}{gray}{0.95}
\lstset{
	% language=Python,
	basicstyle=\footnotesize\ttfamily,
	backgroundcolor=\color[rgb]{0.95, 0.95, 0.95},
	keywordstyle=\color[rgb]{0,0.5,0}\ttfamily,
	stringstyle=\color{orange}\ttfamily,
	commentstyle=\color{red}\ttfamily,
	tabsize=2,
	numbers=left, numberstyle=\tiny, numbersep=5pt,
	breaklines=true,
	breakatwhitespace=true,
	prebreak={/},
	captionpos=b,
	columns=fullflexible,
	escapeinside={\#*}{\^^M},
	mathescape
}
\usepackage{hyperref}

\begin{document}

\begin{titlepage}

\newcommand{\HRule}{\rule{\linewidth}{0.5mm}} % Defines a new command for the horizontal lines, change thickness here

\center % Center everything on the page
 
%----------------------------------------------------------------------------------------
%	HEADING SECTIONS
%----------------------------------------------------------------------------------------

\textsc{\LARGE Uppsala University}\\[1.5cm] % Name of your university/college
\includegraphics[scale=.1]{Uppsala_University_seal_svg.png}\\[1cm] % Include a department/university logo - this will require the graphicx package
\textsc{\Large Secure Coputer Systems I}\\[0.5cm] % Major heading such as course name
\textsc{\large 1DT072}\\[0.5cm] % Minor heading such as course title

%----------------------------------------------------------------------------------------
%	TITLE SECTION
%----------------------------------------------------------------------------------------

\HRule \\[0.4cm]
{ \huge \bfseries Title}\\[0.4cm] % Title of your document
\HRule \\[1.5cm]
 
%----------------------------------------------------------------------------------------
%	AUTHOR SECTION
%----------------------------------------------------------------------------------------

\begin{minipage}{0.4\textwidth}
\begin{flushleft} \large
\emph{Author:}\\
Arvid \textsc{Morelid}\\ % Your name
\end{flushleft}

\end{minipage}\\[2cm]

% If you don't want a supervisor, uncomment the two lines below and remove the section above
%\Large \emph{Author:}\\
%John \textsc{Smith}\\[3cm] % Your name

%----------------------------------------------------------------------------------------
%	DATE SECTION
%----------------------------------------------------------------------------------------

{\large \today}\\[2cm]package % Date, change the \today to a set date if you want to be precise

\vfill % Fill the rest of the page with whitespace

\end{titlepage}

\section*{SQL Injections}
To obtain a list of all users on the site, we utilise the product search page which lists all rows of the returned SQL table. To include the users in the query, which usually is limited to the 'Products' table, we enter the search term as seen in listing \ref{lst:sql}. 
\begin{lstlisting}[caption={Search term to get all users in the database}, label={lst:sql}]
    Nonmatching string')) UNION SELECT *, NULL, NULL, NULL FROM Users --|
\end{lstlisting}


\end{document}